%
% To compile run: 'xelatex CV2018.tex'
%

%--------------------------------------------------------------
%	PACKAGES AND OTHER DOCUMENT CONFIGURATIONS
%--------------------------------------------------------------

\documentclass[a4paper, 10pt]{article}
\usepackage[margin=1.75cm]{geometry}
\title{\bfseries\Large Flavio Toffalini}
\author{}
\date{}

\usepackage{tabto}
\usepackage{hyperref}
  
\usepackage{xspace}
\usepackage{xcolor}
\usepackage{lastpage}
\usepackage{fancyhdr}
\pagestyle{fancy} 
\fancyhf{} % sets both header and footer to nothing
\renewcommand{\headrulewidth}{0pt}
%\cfoot{\thepage~of~2}

\usepackage{array, xcolor}
\definecolor{lightgray}{gray}{0.8}
\newcolumntype{M}{>{\raggedleft}p{0.09\textwidth}}
\newcolumntype{N}{p{0.9\textwidth}}
\newcolumntype{L}{>{\raggedleft}p{0.15\textwidth}}
\newcolumntype{R}{p{0.8\textwidth}}
\newcommand\VRule{\color{lightgray}\vrule width 0.5pt}

%\definecolor{darkred}{rgb}{0.804, 0.365, 0.365}
\definecolor{darkred}{rgb}{0.790, 0.0118, 0.0275}
\definecolor{electron}{rgb}{0.035, 0.518, 0.890}
\definecolor{mint}{rgb}{ 0.000, 0.722, 0.580 }
\definecolor{violet}{rgb}{ 0.424, 0.361, 0.906 }
\definecolor{dblue}{rgb}{ 0.000, 0.300, 0.890 }
\definecolor{darkgray}{gray}{0.45}

\newcommand\darkdot{$\bm{\cdot}$\xspace}
\newcommand{\venue}[2]{\textcolor{electron}{\href{#1}{#2}} $\bm{\cdot}$}%
\newcommand{\coollink}[2]{\textcolor{electron}{\href{#1}{#2}}}%

\newcommand{\desc}[1]{\textcolor{darkgray}{#1}}%

\renewcommand{\familydefault}{\sfdefault}

\newcommand{\ie}{\textit{i.e.,} }
\newcommand{\eg}{\textit{e.g.,} }
\newcommand{\wrt}{\textit{w.r.t.\ }}

\newcommand{\todobox}[3]{%
	\colorbox{#1}{\textcolor{white}{\sffamily\bfseries\scriptsize #2}}%
	~\textcolor{blue}{#3} %
	\textcolor{#1}{$\triangleleft$}%
}
\newcommand{\todo}[1]{\todobox{red}{TODO}{#1}}
\newcommand{\feedback}[1]{\todobox{orange}{OK?}{#1}}
\newcommand{\done}[1]{\todobox{blue}{DONE}{#1}}
\newcommand{\rationale}[1]{\todobox{ForestGreen}{RATIONALE}{#1}}


\usepackage{bm}
\begin{document}

%\thispagestyle{empty}
\begin{center}
\Large Flavio Toffalini\\
\end{center}

%--------------------------------------------------------------
%	ADDRESS AND CONTACT INFORMATION
%--------------------------------------------------------------

\begin{center}
flavio\_toffalini@mymail.sutd.edu.sg\\
+65 8341 0549\\
% \vspace{0.2em}
\end{center}

%--------------------------------------------------------------
%	MAIN CV CONTENT
%--------------------------------------------------------------

% \vspace{.4em}
\subsection*{Objective}
Collaborating with professional teams for overcoming Information Security 
limitations with integrity and ingenuity.
My current research aims to improve DL for malware detection borrowing fresh 
idea from program analysis' world.
Previously, the core of my Ph.D. stretches the security properties of trusted 
execution environments, in particular SGX, by integrating multiple disciplines 
from program analysis, deep learning, network protocols, and cryptography.
The goal of my work is to combine different perspectives in order to design 
strong trustworthy systems for daily tasks.


%--------------------------------------------------------------
% ACADEMIC EXPERIENCE
%--------------------------------------------------------------

\subsection*{Education}

\begin{tabular}{M!{\VRule}N}
	
	2017&{\bf Ph.D. student in Computer Science}\\
	present&{\it Singapore University of Technology and Design (SUTD)}\\
	& Supervisor: Prof. Jianying Zhou; Co-Advisor: Prof. Lorenzo Cavallaro 
	\& Prof. Mauro Conti.\\ 
	& \desc{5th year PhD programme with ISTD pillar studying the limitations of 
		modern trusted execution environments.}\\
	\\
	
	2015&{\bf M.S. in Computer Science and Engineering 108$/$110, GPA 3,9/4}\\
	&{\it University of Verona, IT}\\
	& Supervisor: Prof. Damiano Carra; Co-Advisor: Prof. Davide Balzarotti\\
	& \desc{I wrote my thesis under the supervision of Prof. Carra Damiano and 
		co-advised by Prof. Davide Blazarotti. The topic of my thesis expands 
		the 
		Google Dork project started in Eurecom (FR).
		From the final thesis version, we extracted a research paper entitled 
		``Google Dorks: Analysis, Creation, and new Defenses'' and published in 
		DIMVA 2016.}\\
	\\
	
	2009&{\bf B.S. in Computer Engineer 101$/$110, GPA 3,67/4}\\
	&{\it University of Pavia, IT}\\
	&Supervisor: Prof. Paolo Gamba\\
	& \desc{I wrote my thesis under the supervision of Prof. Paolo Gamba. In 
	the project, I studied new data acquisition techniques for IoT networks.}\\
%	\\
	
\end{tabular}

\subsection*{Visiting Research Scholar}

\begin{tabular}{M!{\VRule}N}
	
	2021&{\bf Research Collaboration}\\
	present&{\it King's College London, UK}\\
	& \desc{I collaborate with Prof. Lorenzo Cavallaro's Systems Security 
		Research Lab on a Multi-Task Learning project. I investigate the 
		challenges 
		of using MTL classifiers for malware analysis. The goal is to detect
		fine-grained malicious behaviors linked to the MITRE ATT\&CK 
		classification.}\\
	\\
	
	Fall 2019&{\bf Visiting Ph.D. Scholar}\\
	&{\it King's College London, UK}\\ 
	& \desc{I collaborated with Prof. Lorenzo Cavallaro's Systems Security 
	Research Lab on a project about the runtime integrity limitations of SGX 
	enclaves. We designed and deployed a runtime remote attestation able to 
	measure and validate the enclave execution-flow. Part of our outcome has 
	been included in the publication ``Revisiting Program Analysis and 
	Intrusion Detection for SGX'' which is currently under submission at CCS 
	2021.} \\
	\\
	
	2018&{\bf Visiting Ph.D. Scholar}\\
	&{\it University of Padua, IT}\\
	& \desc{I collaborated with Prof. Mauro Conti's Security and Privacy 
	Research Group (SPRITZ Group). I studied the limitations of runtime remote 
	attestations in complex software. 
	Part of the outcome has been included in the publication ``ScaRR: Scalable 
	Runtime Remote Attestation for Complex Systems'' and published in RAID 
	2019.}\\
	\\
	
	2015&{\bf Visiting fellow}\\
	&{\it Eurecom, FR}\\
	& \desc{I collaborated with Prof. Davide Balzarotti's System Security team 
	on a Web security project focused on Google Dorks. The objective was the 
	proposal of novel mitigation strategies and study automatic techniques for 
	creating new Google Dorks.}\\	
\end{tabular}

\cfoot{1~of~4}

\subsection*{Employment}

\begin{tabular}{L!{\VRule}R}
	
	2016-2017&{\bf Research Assistant}\\
	&{\it Singapore University of Technology and Design, SG}\\
	& \desc{I worked on an Insider Threat project under the supervision of 
	Prof. Ochoa Mart\`in and co-funded by ST Electronics (SG). The project's 
	purpose was to study novel ML methodologies for detecting insider threats 	
	and developing secure monitoring agents based on trusted computing 		
	technologies.}\\
	\\
	
	2015-2016&{\bf Research Assistant}\\
	&{\it University of Verona, IT}\\
	& \desc{I studied the limitations of static analysis in detecting context 
	memory leaks in Android applications. The project was supervised by Prof. 
	Fausto Spoto and co-funded by Julia Soft.}\\
	\\
	
\end{tabular}

\subsection*{Professional Development}

\begin{tabular}{L!{\VRule}R}
	
	2021&{\bf Deep Learning Specialization}\\
	&{\it Coursera at 
	\coollink{https://www.deeplearning.ai/program/deep-learning-specialization/}{deeplearning.ai}}\\
	& \desc{A six months class that covers many theoretical and practical 
	aspects of Deep Learning: CNN, Transfer Learning, Object 
	Detection,  NLP, GRU, LSTM, Attentions Models, and Multi-Head. 
	Furthermore, the course provided deep knowledge of the Tensorflow 
	framework.}
	\\

\end{tabular}

%--------------------------------------------------------------
% PUBLICATIONS
%--------------------------------------------------------------

\vspace{0.2em}
\subsection*{Selected Publications}

\begin{tabular}{L!{\VRule}R}

2021 &{\bf Following the evidence beyond the wall:	memory forensic in SGX 
environments}\\
&{\it \textbf{F. Toffalini}, A. Oliveri, M. Graziano, J. Zhou, D. Balzarotti}\\
& \venue{}{} under submission \\  
& \desc{We study the impact of memory forensic techniques in SGX machines, 
what information can be extracted and how to use them for analyzing modern SGX 
malware.}\\
\\
&{\bf Revisiting Program Analysis and Intrusion Detection for SGX}\\
&{\it \textbf{F. Toffalini}, J. Zhou, L. Cavallaro}\\ 
& \venue{}{} under submission \\ 
& \desc{We propose a runtime remote attestation that measures the execution 
flow of an SGX enclave and allows a remote Verifier to validate the enclave 
runtime integrity. We test our approach against modern enclave code-reuse 
attacks.} \\
\\
&{\bf SnakeGX: a sneaky attack against SGX Enclaves}\\
&{\it \textbf{F. Toffalini}, M. Graziano, M. Conti, J. Zhou}\\ 
& \venue{https://github.com/tregua87/snakegx/blob/main/paper.pdf}{ACNS} The 
19th International Conference on Applied Cryptography and Network Security 
(ACNS) \\ 
& \desc{We describe a data-only malware able to install a persistent 
backdoor in a running enclave. The backdoor leaves a privilege entrance for 
adversaries that can exfiltrate data in stealthy manner while leaving a limited 
footprint in memory. We compare the traces left with other state of the art 
attacks for SGX.}\\
\\
2019 &{\bf ScaRR: Scalable Runtime Remote Attestation for Complex Systems}\\
&{\it \textbf{F. Toffalini}, E. Losiouk, Biondo A., J. Zhou, M. Conti}\\ 
& 
\venue{https://www.usenix.org/conference/raid2019/presentation/toffalini}{RAID} 
The 22nd International Symposium on Research in Attacks, Intrusions and 
Defenses (RAID) \\ 
& \desc{We describe a model for runtime remote attestation that can represent 
correct executions of complex software in a limited amount of memory. 
We compare our approach with state of the art runtime models and show our 
scalability properties.}\\
\\
&{\bf Insight Into Insiders and IT: A Survey of Insider Threat Taxonomies, 
	Analysis, Modeling, and Countermeasures}\\
&{\it I. Homoliak, \textbf{F. Toffalini}, J. Guarnizo, Y. Elovici, M. 
	Ochoa}\\ 
& \venue{https://dl.acm.org/doi/abs/10.1145/3303771}{CSUR} ACM Computing 
Surveys (CSUR) \\ 
& \desc{We study the state of the art of Insider Threats for IT. We 
	identify 
	common threats, propose a new taxonomy classification, and identify new 
	challenges.}\\
\end{tabular}

\cfoot{2~of~4}

\subsection*{}
\begin{tabular}{M!{\VRule}N}
	
	&{\bf Careful-Packing: A Practical and Scalable Anti-Tampering 
		Software Protection enforced by Trusted Computing}\\
	&{\it \textbf{F. Toffalini}, M. Ochoa, J. Sun, and J. Zhou}\\ 
	& \venue{https://dl.acm.org/doi/abs/10.1145/3292006.3300029}{CODASPY} The 
	9th 
	ACM Conference on Data and Application Security and Privacy (CODASPY)\\ 
	&\desc{We propose a combination of TEE and anti-tampering techniques to 
		strengthen code integrity over untrusted memory regions while using a 
		limited memory in the TEE modules. Our solution introduces a small 
		overhead 
		of around 5\%.}\\
	\\
	2018 &{\bf Static Analysis of Context Leaks in Android Applications}\\
	&{\it \textbf{F. Toffalini}, M. Ochoa, J. Sun, and J. Zhou}\\ 
	& \venue{https://dl.acm.org/doi/abs/10.1145/3183519.3183530}{ICSE} The 40th 
	International Conference on Software Engineering: Software Engineering in 
	Practice (SEPA@ICSE)\\ 
	&\desc{We study the problem of context leak in Android application and 
		propose 
		a simple yet effective static analysis to find such errors.}\\
	\\
	&{\bf Detection of Masqueraders Based on Graph Partitioning of File System 
		Access Events}\\
	&{\it \textbf{F. Toffalini}, I. Homoliak, A. Harilal, A. Binder, and M. 
		Ochoa} \\ & 	
		\venue{https://www.computer.org/csdl/proceedings-article/sp/2018/435301a964/12OmNC1Gugf}{SPW}
	IEEE Security and Privacy Workshops (SPW) \\ 
	&\desc{We propose a novel detection, based on graph comparison, that spots 
		anomaly activities in sequence of events. We apply our approach over 
		two open 
		source datasets, WUIL and TWOS, achieving an AUC over 0,94 and 0.85, 
		respectively.}\\
	\\
	2017&{\bf TWOS: A Dataset of Malicious Insider Threat Behavior Based on a 
		Gamified Competition}\\
	&{\it A. Harilal, \textbf{F. Toffalini}, J. Castellanos, J. Guarnizo, I. 
		Homoliak,  and M. Ochoa}\\ 
	& 
	\venue{https://dl.acm.org/doi/10.1145/3139923.3139929}{MIST}
	The 9th ACM CCS International Workshop on Managing Insider Security Threats 
	(MIST) \\ 
	&\desc{We present a dataset of insider threats obtained through a gamified 
		competition that involved 6 teams of 4 students and that lasted for 
		more than 
		300 hours. We designed the game in order to simulate realistic insider 
		threats 
		scenarios.}\\
	\\
	2016&{\bf Google Dorks: Analysis, Creation, and new Defenses}\\
	&{\it \textbf{F. Toffalini}, M. Abba', D. Carra, and D. Balzarotti}\\ 
	& 
	\venue{https://link.springer.com/chapter/10.1007/978-3-319-40667-1_13}{DIMVA}
	Detection of Intrusions and Malware, and Vulnerability Assessment The 13th 
	International Conference, (DIMVA)\\ 
	&\desc{We study the impact of existing Google Dorks and proposed 
		mitigation. 
		In addition, we study new techniques to automatically generate new 
		Google 
		Dorks and propose relative mitigation.}\\
	
\end{tabular}
%\cfoot{2~of~3}
%--------------------------------------------------------------
% WORK EXPERIENCE
%--------------------------------------------------------------

%\vspace{0.0em}
%\subsection*{Industry Experience}
%\begin{tabular}{L!{\VRule}R}
%
%Fall 2020&{\bf PhD Software Engineering Intern}\\
%&{\it Facebook}\\
%& \desc{Worked with the Compromised Accounts Measurement team to develop a 
%system for automatically generating high precision policies to bootstrap 
%labeling for large-scale compromised account detection and measurement tasks. 
%Awarded a returning offer for 2021.}\\
%\\
%\end{tabular}

%--------------------------------------------------------------
%	REFERENCES
%--------------------------------------------------------------

%\vspace{2.4em} 
%\textit{References available upon request}.
\subsubsection*{References available upon request}
\vspace{-0.3em}
\begin{itemize}
	\setlength{\itemsep}{1pt}
	\setlength{\parskip}{0pt}
	\setlength{\parsep}{0pt}
	
	\TabPositions{6.5cm}
	\item Prof. Janying Zhou (Ph.D. Supervisor) \tab jianying\_zhou@sutd.edu.sg
	\item Prof. Lorenzo Cavallaro (Ph.D co-advisor) \tab 
	lorenzo.cavallaro@kcl.ac.uk
	\item Prof. Mauro Conti (Ph.D. co-advisor) \tab mauro.conti@unipd.it
	\item Prof. Davide Balzarotti (co-author) \tab davide.balzarotti@eurecom.fr
	\item Mariano Graziano (co-author) \tab magrazia@cisco.com
\end{itemize}

%--------------------------------------------------------------
%	VOLUNTEERING
%--------------------------------------------------------------

%--------------------------------------------------------------
%	SERVICE
%--------------------------------------------------------------

%\vspace{-0.35em}

%\vspace{0.06em}
%\subsection*{Community Service}
%
%\vspace{-1.38em}
\subsubsection*{PC Member}
\vspace{-0.3em}
\begin{itemize}
	\setlength{\itemsep}{1pt}
	\setlength{\parskip}{0pt}
	\setlength{\parsep}{0pt}
  
	\TabPositions{5cm}
	\item IEEE S\&P (Shadow PC) \tab 2021
	\item SecMT \tab 2020, 2021
\end{itemize}

%
%\subsubsection*{Committee}
%\vspace{-0.3em}
%\begin{itemize}
%  \setlength{\itemsep}{1pt}
%  \setlength{\parskip}{0pt}
%  \setlength{\parsep}{0pt}
%  
%\TabPositions{4.3cm}
%\end{itemize}

%
\vspace{-1.38em}
\subsubsection*{Reviewer}
\vspace{-0.3em}
\begin{itemize}
  \setlength{\itemsep}{1pt}
  \setlength{\parskip}{0pt}
  \setlength{\parsep}{0pt}
  
\TabPositions{5cm}
\item EuroSec (subreviewer) \tab 2021
\item RAID (subreviewer) \tab 2020
\item USENIX Security (subreviewer)\tab 2020
\item ESORICS (subreviewer) \tab 2018
\item TIFS (subreviewer) \tab 2018, 2019
\end{itemize}

%\vspace{-1.38em}
\subsubsection*{Conference Volunteer}
\vspace{-0.3em}
\begin{itemize}
  \setlength{\itemsep}{1pt}
  \setlength{\parskip}{0pt}
  \setlength{\parsep}{0pt}
  
\TabPositions{5cm}
\item ACM CCS 				\tab 2019
\end{itemize}

\cfoot{3~of~4}

\subsubsection*{Teaching Assistant}
\vspace{-0.3em}
\begin{itemize}
  \setlength{\itemsep}{1pt}
  \setlength{\parskip}{0pt}
  \setlength{\parsep}{0pt}
  
\TabPositions{9cm, 10.5cm}
\item Cyber Challenge Seminars \tab UNIPD \tab 2020
\item 50.039 Theory and Practice of Deep Learning \tab SUTD \tab 2019
\item 50.005 Computer System Engineering \tab SUTD \tab 2019
\item Informatics and Bioinformatics \tab UNIPD \tab 2018
\end{itemize}

%\vspace{-1.38em}
\subsubsection*{Other}
\vspace{-0.3em}
\begin{itemize}
  \setlength{\itemsep}{1pt}
  \setlength{\parskip}{0pt}
  \setlength{\parsep}{0pt}

%\item {\bf Founding team member} of both RHUL and KCL capture-the-flag 
%competition teams. 
\item {\bf MSc Co-supervision}: Mohamad Ridzuan Yusop, rebuilding ROP chains 
from Intel PT traces.
\item {\bf MSc Co-supervision}: Jon Kartago Lamida, comparing modern runtime 
remote attestations models. 
\item {\bf MSc Co-supervision}: Alessandro Visintin, design novel remote 
attestation schemes for networks of IoT devices.
\end{itemize}


\cfoot{4~of~4}

%--------------------------------------------------------------
%	LANGUAGES
%--------------------------------------------------------------

\end{document}




